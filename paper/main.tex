\documentclass[12pt]{article}

% --- Packages ---
\usepackage{amsmath, amssymb, amsfonts}
\usepackage{graphicx}
\usepackage{geometry}
\usepackage{hyperref}
\usepackage{authblk}
\usepackage{tikz}
\usepackage{physics}
\usepackage{enumitem}
\usepackage{mathtools}
\geometry{margin=1in}

% --- Title ---
\title{Unified Angular Mathematics: Cosmological Geometry and the $\pi^3$ Paradigm}
\author{Albert Pacelli}
\date{\today}

\begin{document}
\maketitle

% --- Abstract ---
\begin{abstract}
Unified Angular Mathematics (UAM) proposes that angular bifurcation and interference geometry play foundational roles in early-universe energy partitioning. This framework investigates the appearance of $\pi^3$ in cosmological ratios and the possibility that angular constructs underlie matter-energy densities and scalar field evolution. Here we outline initial theoretical structures, simulation pipelines, and empirical targets in pursuit of a falsifiable UAM model.
\end{abstract}

% --- Table of Contents ---
\tableofcontents
\vspace{1cm}

% --- Section Scaffold ---
\section{Introduction}
\begin{itemize}[leftmargin=*, label={--}]
  \item Motivation for UAM as a geometric-first framework
  \item Prior appearances of $\pi^3$ in cosmology (e.g., density ratios)
  \item Limitations of existing partition models
\end{itemize}

\section{Angular Bifurcation and Energy Dispersion}
\begin{itemize}[leftmargin=*, label={--}]
  \item Definition and formalism for angular bifurcation
  \item Sphere wedge geometry and logical chord constructs
  \item Hypothesis: Bifurcation’s role in scalar field symmetry-breaking
\end{itemize}

\section{Interference Geometry and Cosmological Structure}
\begin{itemize}[leftmargin=*, label={--}]
  \item Modeling interference patterns and multipole modes
  \item Relationships between $\theta$ evolution and $H(z)$
  \item Early wavefront partitioning simulations
\end{itemize}

\section{Numerical Simulations}
\begin{itemize}[leftmargin=*, label={--}]
  \item Python-based MCMC setup and parameter estimation
  \item Alignment with Planck, BAO, Pantheon+ datasets
  \item Recovery of $\pi^3/100$ and $\Omega_\Lambda/\Omega_m$ signatures
\end{itemize}

\section{Discussion and Future Work}
\begin{itemize}[leftmargin=*, label={--}]
  \item Potential falsifiability tests
  \item Predictions beyond the Standard Model (e.g., decay channels)
  \item Pathways toward unified field constructs
\end{itemize}

% --- References ---
\bibliographystyle{plain}
\bibliography{references}  % If you create a 'references.bib'

\end{document}
